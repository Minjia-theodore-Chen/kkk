\setcounter{section}{0}
\part{Fahrzeugsystemtechnik allegemain}
\section{Fahrzeugsystemtechnik}
\subsection{}\textbf{Was versteht man unter dem Begriff der Emergenz?}

Emergenz ist das Prinzip, dass Systeme wichtige Eigenschaften (z.B. Komfort, Sicherheit) nur aufweisen, wenn sie auf das ganze System angewendet werden und nicht auf einzelne Teile.

\section{Systemtheorie}
\subsection{}
\textbf{Welche Bedingungen müssen erfüllt sein, um ein System als linear zu klassifizieren? Geben Sie auch eine mathematische Formulierung der nötigen Bedingungen an.}

Ein System ist linear, wenn die Bedingungen der Linearität und der Additivität erfüllt sind. Ist eine der Bedingungen nicht erfüllt ist, ist das System nicht linear!
\begin{itemize}
    \item Linearit\"at:Vergrößerung des Eingangssignal um eine beliebige Konstante c vergrößert auch das Ausgangssignal um die Konstante c.
          \begin{equation}
              \begin{aligned}
                  y_1        & =S\{u_1\}        \\
                  c\cdot y_1 & =S\{c\cdot u_1\} \\
              \end{aligned}
          \end{equation}
    \item Additivit\"at: Die Summe zweier Signale am Eingang ergibt das gleiche Ausgangssignal wie wenn die beiden einzelnen Ausgangssignale addiert werden.
          \begin{equation}
              \begin{aligned}
                  y_1     & =S\{u_1\}          \\
                  y_2     & =S\{u_2\}          \\
                  y_1+y_2 & =S\{u_1\}+S\{u_2\} \\
              \end{aligned}
          \end{equation}
\end{itemize}
Zusammen:
\begin{equation}
    c_1\cdot y_1+c_2\cdot y_2=S\{c_1\cdot u_1+c_2\cdot u_2\}
\end{equation}.

\subsection{}
\textbf{Wie ist Übertragungsstabilität für ein zeitkontinuierliches System definiert (qualitativ)?}

\subsection{}
\textbf{Geben Sie die allgemeine vektorielle Darstellung der Zustandsgleichung für die Systembeschreibung an.}

\subsection{}
\textbf{Erläutern Sie qualitativ welches Wissen über das System in welchen Matrizen modelliert wird.}

\subsection{}
\textbf{Wie lässt sich aus der Zustandsraumdarstellung die Übertragungsfunktion eines Systems ermitteln? Geben Sie hierbei auch Ihre Herleitung an.}

\subsection{}
\textbf{Wie ist die Zustandsstabilität eines Systems qualitativ definiert? Geben Sie zusätzlich ein mathematisches Kriterium an.}

\subsection{}
\textbf{Nennen Sie ein mathematisches Kriterium zum Prüfen der Steuerbarkeit.}