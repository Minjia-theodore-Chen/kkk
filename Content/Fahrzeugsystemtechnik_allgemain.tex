\setcounter{section}{0}
\part{Fahrzeugsystemtechnik allegemain}
\section{Fahrzeugsystemtechnik}
\subsection{}\textbf{Was versteht man unter dem Begriff der Emergenz?}

Emergenz ist das Prinzip, dass Systeme wichtige Eigenschaften (z.B. Komfort, Sicherheit) nur aufweisen, wenn sie auf das ganze System angewendet werden und nicht auf einzelne Teile.

\section{Systemtheorie}
\subsection{}
\textbf{Welche Bedingungen müssen erfüllt sein, um ein System als linear zu klassifizieren? Geben Sie auch eine mathematische Formulierung der nötigen Bedingungen an.}

Ein System ist linear, wenn die Bedingungen der Linearität und der Additivität erfüllt sind. Ist eine der Bedingungen nicht erfüllt ist, ist das System nicht linear!
\begin{itemize}
    \item Linearit\"at:Vergrößerung des Eingangssignal um eine beliebige Konstante c vergrößert auch das Ausgangssignal um die Konstante c.
          \begin{equation}
              \begin{aligned}
                  y_1        & =S\{u_1\}        \\
                  c\cdot y_1 & =S\{c\cdot u_1\} \\
              \end{aligned}
          \end{equation}
    \item Additivit\"at: Die Summe zweier Signale am Eingang ergibt das gleiche Ausgangssignal wie wenn die beiden einzelnen Ausgangssignale addiert werden.
          \begin{equation}
              \begin{aligned}
                  y_1     & =S\{u_1\}          \\
                  y_2     & =S\{u_2\}          \\
                  y_1+y_2 & =S\{u_1\}+S\{u_2\} \\
              \end{aligned}
          \end{equation}
\end{itemize}
Zusammen:
\begin{equation}
    c_1\cdot y_1+c_2\cdot y_2=S\{c_1\cdot u_1+c_2\cdot u_2\}
\end{equation}.

\subsection{}
\textbf{Wie ist Übertragungsstabilität für ein zeitkontinuierliches System definiert (qualitativ)?}

Ein zeitkontinuierliches System ist übertragungsstabil, wenn es auf jede beschränkte Eingangsgröße mit einer beschränkten Ausgangsgröße antwortet.

Sonst heißt das System übertragungsinstabil!

Übertragungsstabilität wird deshalb auch BIBO-Stabilität (Bounded Input – Bounded Output) genannt.

\textbf{Übertragungsfunktions-Kriterium zur Ermittlung der Übertragungsstabilität:}

Ein LZI-System mit der rationalen Übertragungsfunktion ist übertragungsstabil, wenn alle Pole seiner Übertragungsfunktion in der linken s-Halbebene liegen.
\subsection{}
\textbf{Geben Sie die allgemeine vektorielle Darstellung der Zustandsgleichung für die Systembeschreibung an.}

\begin{equation}
    \begin{aligned}
        \left[
            \begin{array}{c}
                \dot{x_1}(t) \\
                \cdot        \\
                \cdot        \\
                \cdot        \\
                \dot{x_n}(t) \\
            \end{array}
        \right] & =\underbrace{\left[
                \begin{array}{ccccc}
                    a_{11} & \cdot & \cdot & \cdot & a_{1n} \\
                    \cdot  & \cdot & \cdot & \cdot & \cdot  \\
                    \cdot  & \cdot & \cdot & \cdot & \cdot  \\
                    \cdot  & \cdot & \cdot & \cdot & \cdot  \\
                    a_{n1} & \cdot & \cdot & \cdot & a_{nn} \\
                \end{array}
                \right]}_{systemmatrix}\underbrace{\left[
                \begin{array}{c}
                    x_1(t) \\
                    \cdot  \\
                    \cdot  \\
                    \cdot  \\
                    x_n(t) \\
                \end{array}
                \right]}_{Zustandsvektor}+\underbrace{\left[
                \begin{array}{c}
                    b_1   \\
                    \cdot \\
                    \cdot \\
                    \cdot \\
                    b_n   \\
                \end{array}
                \right]}_{Eingangsvektor}
        \cdot u(t)                                                                                 \\
        y(t)    & =\underbrace{\left[\begin{array}{ccccc}
                    c_1 & \cdot & \cdot & \cdot & c_n
                \end{array}\right]}_{Ausgangsvektor}\cdot\left[
            \begin{array}{c}
                x_1(t) \\
                \cdot  \\
                \cdot  \\
                \cdot  \\
                x_n(t) \\
            \end{array}
            \right]+\underbrace{d}_{Durchgangsfaktor(meist=0)}\cdot u(t)
    \end{aligned}
\end{equation}

\subsection{}
\textbf{Erläutern Sie qualitativ welches Wissen über das System in welchen Matrizen modelliert wird.}

\subsection{}
\textbf{Wie lässt sich aus der Zustandsraumdarstellung die Übertragungsfunktion eines Systems ermitteln? Geben Sie hierbei auch Ihre Herleitung an.}

\begin{equation}
    \begin{aligned}
        \dot{x}(t)             & =A\cdot x(t)+b\cdot u(t)\qquad Annahme x(0)=0                                    \\
        sX(s)                  & =A\cdot X(s)+b\cdot u(s)                                                         \\
        sX(s)-A\cdot X(s)      & =b\cdot u(s)                                                                     \\
        (s\cdot I-A)\cdot X(s) & =b\cdot u(s)                                                                     \\
        X(s)                   & =(S\cdot I-A)^{-1}\cdot b\cdot u(s)                                              \\
        y(t)                   & =C^T\cdot x(t)+d\cdot u(t)                                                       \\
        Y(s)                   & =C^T\cdot X(s)+d\cdot u(s)                                                       \\
        Y(s)                   & =\underbrace{(C^T(S\cdot I-A)^{-1}\cdot b+d)}_{F(s)\ allgemeiner Fall}\cdot u(s)
    \end{aligned}
\end{equation}
\subsection{}
\textbf{Wie ist die Zustandsstabilität eines Systems qualitativ definiert? Geben Sie zusätzlich ein mathematisches Kriterium an.}

Ein System ist zustandsstabil (oder asymptotisch stabil nach Ljapunow), wenn:
\begin{itemize}
    \item es in seiner Ruhelage bleibt, solange es nicht von außen angeregt wird und
    \item es in seine Ruhelage zurückkehrt, wenn alle äußeren Wirkungen von ihm weggenommen werden
\end{itemize}
Das System, beschrieben durch die Zustandsgleichungen
\begin{equation}
    \begin{aligned}
        \dot{\underline{x}}(t) & =\underline{A}\cdot\underline{x}(t)+\underline{b}\cdot\underline{u}(t)   \\
        \underline{y}(t)       & =\underline{c}^T\cdot\underline{x}(t)+\underline{d}\cdot\underline{u}(t) \\
    \end{aligned}
\end{equation}
ist genau dann zustandsstabil, wenn alle n Eigenwerte $\lambda_1,\cdots,\lambda_n$ der Matrix A links der imaginären Achse
der komplexen s-Ebene liegen, also alle Realteile negativ sind.

\subsection{}
\textbf{Nennen Sie ein mathematisches Kriterium zum Prüfen der Steuerbarkeit.}

Kalman-Kriterium zur Steuerbarkeit:
Das System, beschrieben durch die Zustandsgleichungen
\begin{equation}
    \begin{aligned}
        \dot{\underline{x}}(t) & =\underline{A}\cdot\underline{x}(t)+\underline{b}\cdot\underline{u}(t)   \\
        \underline{y}(t)       & =\underline{c}^T\cdot\underline{x}(t)+\underline{d}\cdot\underline{u}(t) \\
    \end{aligned}
\end{equation}
ist genau dann vollständig steuerbar, wenn seine$(n\times n)$ -Steuerbarkeitsmatrix
\begin{equation}
    \underline{Q}_S=\left(\underline{b}\ \underline{Ab}\ \underline{A^2b}\cdots\underline{A^{n-1}b}\right)
\end{equation}
regulär ist, d.h. $\detmat\left(\underline{Q}_S\right)\neq 0$ gilt.