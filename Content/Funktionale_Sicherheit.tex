\part{Funktionale Sicherheit}
\section{ISO26262}
\subsection{}
\textbf{Im Rahmen der Vorlesung haben Sie die Funktionale Sicherheitsbetrachtung nach ISO 26262 kennengelernt.}

\textbf{Nennen Sie die Definition von Funktionaler Sicherheit nach ISO 26262}

Funktionale Sicherheit ist die Abwesenheit unzumutbarer Risiken aufgrund von Gefährdungen, verursacht durch ein fehlerhaftes Verhalten eines E/E-Systems.

\subsection{}
\textbf{Im Rahmen der Vorlesung haben Sie die Funktionale Sicherheitsbetrachtung nach ISO 26262 kennengelernt.}

\textbf{Welche beiden generellen Fehlertypen können zum Ausfall eines Systems gemä\ss der ISO 26262 führen?}
\begin{itemize}
    \item Systematische Fehler
    \item Zufällige Hardwarefehler
\end{itemize}

\subsection{}
\textbf{Im Rahmen der Vorlesung haben Sie die Funktionale Sicherheitsbetrachtung nach ISO 26262 kennengelernt.}

\textbf{Was wird in der Item Definition festgelegt?}

Definition von:
\begin{itemize}
    \item Systemanforderungen(Funktionalität)
    \item Systemgrenzen
    \item Abhängigkeiten und Schnittstellen
\end{itemize}

\subsection{}
\textbf{Im Rahmen der Vorlesung haben Sie die Funktionale Sicherheitsbetrachtung nach ISO 26262 kennengelernt.}

\textbf{Was sind die drei zentralen Dinflussfaktoren für eine Gefährdungsanalyse und Risokobewertung (G\&R) nach ISO 26262?}

\begin{equation}
    \begin{array}{cc}
        R & =F(f,C,S)       \\
        f & =\lambda\cdot E \\
    \end{array}
\end{equation}
$R$: Risiko

$f$: Auftrentensfrequenz

$C$: Beherrschbarkeit

$S$: Mögliche Schadenschwere

$\lambda$: Ausfallrate des Systems

$E$: Auftrentenswahrscheinlichkeit der Fahrsituation

Vereinfachte Annahme der ISO 26262: $\underline{f=E}$

\subsection{}
\textbf{Welche Defizite besizt die ISO 26262 im Bezug auf die Entwicklung automatisierter Fahrfunktionen?}

\textbf{Randbedingungen} Fahrerassistenz/ Automatisiertes Fahren
\begin{itemize}
    \item Systeme mit Umfeldwahrnehmung und -interpretation
    \item Fahrzeug bewegt sich in einer offenen Umgebung
    \item Nicht quantifizierbare Menge an Szenarien
\end{itemize}

\section{Sicherheit als ermergente Eigenschaft}
\subsection{}
\textbf{Sie haben in der Vorlesung eine Definition von verschiedenen Sichten auf die Sicherheit kennengelernt wie sie etwa von Waymo verwendet wird.}

\textbf{Was versteht man in diesem Zusammenhang unter Verhaltenssicherheit und was unter Betriebssicherheit?}