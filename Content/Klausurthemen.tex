\part{M\"ogliche Klasurthemen}
\section{Vorlesung}
\begin{itemize}
    \item Fahrzeugsystemtechnik\begin{itemize}
              \item Perspektiven in der Fahrzeugentwicklung
              \item Allegemaines(Komplizit\"at, Emergenz,...)
              \item Systemtheorie
          \end{itemize}
    \item Methoden zur Beherrschung von Komplizit\"at\begin{itemize}
              \item Entwicklungsprozess\begin{itemize}
                        \item Vorgehensmodelle\begin{itemize}
                                  \item Phasenmodelle(Wasserfall, Software-Lebenszyklus, V-Modell '97, V-Modell XT, Evolution\"are Modelle)
                                  \item Entwurfsmodelle(Systematischer Entwurf, +Erweiterung)
                              \end{itemize}
                    \end{itemize}
          \end{itemize}
    \item Architekturen\begin{itemize}
              \item Allegemeines
              \item Hierarchische System
              \item Verhaltensbasiert(Subsumption, Rasmussen,Donges, 4D)
              \item Nutzung im Entwicklungsprozess
          \end{itemize}
    \item Modellbildung\begin{itemize}
              \item Beschreibungsebenen/BEgriffe
              \item R\"aumlich-Zeitliche Modelle(Lineare kontinuierliche Systeme, Lineares Einspurmodell, Quer-\linebreak f\"uhrungsmodell 5. \& 3. Ordnung, Beobachter)
              \item Einfache Zustandsregelung
              \item Diskrete ereignisorientierte Modelle(Automaten, Zustandskaten)
          \end{itemize}
\end{itemize}
\section{\"Ubung}
\begin{itemize}
    \item UML\begin{itemize}
              \item Modellierung eines Systems mittels\begin{itemize}
                        \item Aktivit\"atsdiagramm
                        \item Zustandsdiagramm
                    \end{itemize}
          \end{itemize}
\end{itemize}