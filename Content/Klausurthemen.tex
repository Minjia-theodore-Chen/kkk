\part{M\"ogliche Klasurthemen}
\section{Vorlesung}
\begin{itemize}
    \item Fahrzeugsystemtechnik\begin{itemize}
              \item Perspektiven in der Fahrzeugentwicklung
              \item Allegemaines(Komplizit\"at, Emergenz,...)
              \item Systemtheorie
          \end{itemize}
    \item Methoden zur Beherrschung von Komplizit\"at\begin{itemize}
              \item Entwicklungsprozess\begin{itemize}
                        \item Vorgehensmodelle\begin{itemize}
                                  \item Phasenmodelle(Wasserfall, Software-Lebenszyklus, V-Modell '97, V-Modell XT, Evolution\"are Modelle)
                                  \item Entwurfsmodelle(Systematischer Entwurf, +Erweiterung)
                              \end{itemize}
                    \end{itemize}
          \end{itemize}
    \item Architekturen\begin{itemize}
              \item Allegemeines
              \item Hierarchische System
              \item Verhaltensbasiert(Subsumption, Rasmussen,Donges, 4D)
              \item Nutzung im Entwicklungsprozess
          \end{itemize}
    \item Modellbildung\begin{itemize}
              \item Beschreibungsebenen/BEgriffe
              \item R\"aumlich-Zeitliche Modelle(Lineare kontinuierliche Systeme, Lineares Einspurmodell, Quer-\linebreak f\"uhrungsmodell 5. \& 3. Ordnung, Beobachter)
              \item Einfache Zustandsregelung
              \item Diskrete ereignisorientierte Modelle(Automaten, Zustandskaten)
          \end{itemize}
    \item Sicherheit als emergente Eigenschaft\begin{itemize}
        \item ISO26262\begin{itemize}
            \item Konzeptphase (Item Definition, G \& R, FuSiKo)
            \item \textcolor{gray}{Management der FuSi}
            \item \textcolor{gray}{Unterstützende Prozesse}
            \item Defizite
        \end{itemize}
    \end{itemize}
    \item Test\begin{itemize}
        \item Testen\begin{itemize}
            \item Grundlagen (Vokabular, Testaufbau, Testablauf, Testteam)
            \item Testarten (Allgemein, Klassifikation von Prüftechniken, Black-Box-Test, White-Box-Test)
            \item Szenarienbasiertes Testen
        \end{itemize}
    \end{itemize}
\end{itemize}

\section{\"Ubung}
\begin{itemize}
    \item UML\begin{itemize}
              \item Modellierung eines Systems mittels\begin{itemize}
                        \item Aktivit\"atsdiagramm
                        \item Zustandsdiagramm
                        \item
                    \end{itemize}
          \end{itemize}
    \item Systemtheorie\begin{itemize}
              \item Beobachtbarkeit
              \item Steuerbarkeit
              \item Stabilitätskriterien
              \item LZI-System
          \end{itemize}
    \item Einspurmodell\begin{itemize}
              \item Wichtige Größen kennen
              \item Annahmen und Auswirkungen
              \item Schwimmwinke
              \item Querführungsmodell reduzieren 5. -$>$ 2.
          \end{itemize}
    \item Beobachter\begin{itemize}
        \item Idee
        \item Bedingung für Konstruktion
        \item Blockschaltbild
        \item Luenberger-Matrix L berechnen können
    \end{itemize}
    \item Zustandsautomaten\begin{itemize}
        \item Moore- \& Mealy-Automaten
        \item Erweiterung durch Zustandskarten (State Charts)
    \end{itemize}
\end{itemize}