\part{Testen}
\section{Fachtermini}
\textbf{Für das deutsche Wort "Fehler" gibt es im Bereich des Testens verschiedene Fachtermini für unterschiedliche Situationen. Geben Sie für jede der Lücken einen deutschen \underline{und} englischen Fachterminius an.}
\begin{enumerate}
    \item \textbf{Der Programmierer begeht einen} \underline{Fehlhandlung (error)} ...
    \item \textbf{... und hinterlässt einen} \underline{Fehlerzustand (bug)} \textbf{im Programmcode.}
    \item \textbf{Wird dieser ausgeführt, tritt ein} \underline{Fehlerzustand (fault)} \textbf{im Programmzustand auf,}
    \item \textbf{der sich als eine} \underline{Fehlerwirkung (failure)} \textbf{nach au\ss en manifestiert.}
\end{enumerate}

\section{}
\textbf{Was lässt sich durch Testen nachweisen, was hingegen nicht?}

\section{Ziel}
\textbf{Was sind die Ziele des Testens?}

\section{Testteam}
\textbf{Welche Rollen im Testteam unterscheidet man? Welche Aufgaben sind mit der jeweiligen Rolle verbundenen?}

\section{Testarten}
\textbf{Welche vier Testarten bzgl. des V-Modells von '97 haben Sie in der Vorlesung kennengelernt? Was zeichnet diese Testarten jeweils aus?}
\begin{enumerate}
    \item Modultest\begin{itemize}
              \item Isolierte Überprüfung der einzelnen Module
              \item Prüfung einzelner Methoden und Klassen
              \item Prüfung der fehlerfreien Funktion eines Models bezogen auf die Modulspezifikation
          \end{itemize}
    \item Integrationstest\begin{itemize}
              \item Testen mehrer Module im Verbund oder Subsysteme
              \item Schrittweises Zusammenfügen zum Gesamtsystem
              \item Schwerpunkt liegt auf Prüfung des korrekten Zusammenwirkens der Module
          \end{itemize}
    \item Systemtest\begin{itemize}
              \item Testen des vollständigen Systems
              \item Überprüfung der Funktionalität und Leistung gegen die Spezifikation(beim Hersteller/Entwickler)
          \end{itemize}
    \item Abnahmetest\begin{itemize}
        \item Nicht: Fehler hinsichtlich der Spezifikation finden
        \item Sondern: prüfen, ob das System \textbf{aus Kundensicht} die vereinbarten Leistungsmerkmale aufweist
    \end{itemize}
    $\rightarrow$ Validierung
\end{enumerate}
\section{Verifikation und Validierung}
\textbf{Worin besteht der Unterschied zwischen der Verifikation und der Validierung?}
\begin{table}[H]
    \centering
    \begin{tabular}{p{.45\linewidth}p{.45\linewidth}}
        \toprule
        Verifikation                                                                                          & Validierung                                                                                                                                      \\
        \midrule
        Verifikation stellt den Prozess dar, der beurteilt, ob ein System richtig entwickelt wurde            & Validierung beschreibt den Prozess, der überprüft, ob das richtige System entwickelt wurde                                                       \\
        Die Verifikation ist eine Prüfung mit objektiven Mitteln, ob spezifizierte Anforderungen erfüllt sind & Die Validierung ist eine Prüfung mit objektiven Mitteln, ob ein Produkt für eine bestimmte Anwendung oder einen bestimmten Gebrauch geeignet ist \\
        \bottomrule
    \end{tabular}
\end{table}

\section{White- und Black-Box-Testverfahren}
\textbf{Nennen Sie jeweils drei Eigenschaften von White-Box- und Black-Box-Testverfahren.}
\begin{table}[H]
    \centering
    \begin{tabular}{p{.45\linewidth}p{.45\linewidth}}
        \toprule
        Black-Box-Tests                                          & White-Box-Tests                                      \\
        \midrule
        Testfälle aus Spezifikation                              & Testfälle ausgehend von Struktur des Testgegenstands \\
        Innere Struktur bei Ermittlung der Testfälle unbekannt   & Testfälle auch vom Entwickler                        \\
        Testendekriterienanhand der spezifizierten Anforderungen & Testendekriterienanhand des Codes                    \\
        \bottomrule
    \end{tabular}
\end{table}





